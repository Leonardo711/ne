The explosion in number and variety of mobile applications (apps) has made app discovery and selection a great challenge. In order to stand out from numerous apps, developers tend to employ existing third-party libraries. A mobile app with good libraries can provide much better user experience subsequently leading to higher user ratings and more downloads. Based on the statistical analysis of existing mobile apps, this paper observes the fact that mobile apps with higher user ratings tend to employ more third-party libraries. The user ratings represent the popularity and quality of mobile apps, which can have great impact on the revenue. It is one of the most significant aspects that developers care about. In order to help developers make full use of the available third-party libraries, this paper proposes a recommendation approach called \textbf{Lib}rary \textbf{Rec}ommendation based on \textbf{T}extual and \textbf{V}iew similarities (TV-LibRec). This paper presents two similarity computation methods from textual and view perspectives respectively, to identify the similarity between two apps. From the textual perspective, this paper utilizes the description information from metadata of mobile apps. Firstly, this paper applies the data cleaning method to the description by removing punctuation marks, hyperlinks, stop words, \textit{etc.}. Secondly, the cleaned description is converted into vector representation, which is subsequently used to compute the similarity between different descriptions using Pearson Correlation Coefficient and Cosine Similarity methods. From the view perspective, this paper makes full use of two types of view information in Android mobile apps, \textit{i.e.}, resources and assets, and extracts view features from corresponding source codes as well as the declaration of permissions. The similarity between different mobile apps based on their view information is computed by Jaccard index. The textual and view similarities are subsequently used to cluster apps with the proposed \textit{k-min} clustering method. It fully leverages the relationship between mobile apps, and can maximize the whole similarity in each cluster. The method effectively clusters similar apps into the same group and surpasses other existing well-known clustering algorithms. Based on the clustering results, this paper takes both functionality and quality of mobile apps into consideration, \textit{i.e.}, the usage of third-party libraries and the user ratings, so as to recommend high-quality and suitable libraries for developers. Comprehensive experiments are conducted on 23,398 real Android apps crawled from the Google Play store. The experimental results show that the proposed approach outperforms other existing well-known methods, and can give a good recommendation for developers.