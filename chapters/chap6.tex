\chapter{总结与展望}

\section{工作总结}
飞速发展的互联网影响了各行各业,与此同时越来越多的网络型数据积累下来,相应地,网络结构数据涉及的领域也越来越广,不同领域的应用都少不了对网络类数据的挖掘,比如社交网络、,金融交易网络、生物蛋白质网络,这些网络都是用来表示一个个复杂系统,对于网络数据的挖掘和分析有着巨大的机遇和挑战。近年来,图表征算法引起了研究人员的兴趣,不同于传统图挖掘中的社区发现、链路预测任务,图表征算法是基于网络结构或附加属性对网络中的节点学习一个低维向量表征,在获取节点低维向量表征后,可用于后续的机器学习任务。用于聚类,可以实现传统的社区发现任务,用于分类可以用来做链路预测等。现有的图表征算法大部分应用于离线场景,对于新增数据节点无法快速的进行表征,需要重新对全图进行表征学习,而实际应用中,网络数据是处在不断地变化中的,因此在实际应用场景中需要增量的图表征算法,对于新增节点进行快速的表征学习,本文主要研究增量场景下图表征算法,并通过实验来验证算法的有效性。本文的主要工作和相关结论总结如下:
\begin{itemize}
	\item 分析目前业界的图表征算法,基于接近度的概念对拉普拉斯特征映射进行优化,提出了一种保留高阶接近度的图表征算法HLE(High-order Laplacian Eigenmaps),通过对节点进行不同方法的相似度计算得到表示高阶接近度的矩阵,来改进低阶接近度在节点分类和链路预测准确率不高的缺点。
	\item 将HLE算法应用于增量更新的网络之中,提出增量学习算法iHLE,通过对网络增量成分的分析对原表征结果进行调整,在保证后续学习任务准确率的同时提升了运行的时间效率。
	\item 考虑到现实场景中网络数据有丰富的节点属性,分析网络中节点属性和网络结果表征的差异,本文提出了一种基于相似度计算的节点属性表征算法SR(Similarity Representation),并在此基础上提出在增量场景下表征算法iSR,在真实数据实验,表现出良好的准确率和时间复杂度。		
	\item 本文在进行增量实验的抽样过程时采用基于节点重要性逆序采样的方式简化抽样过程;在链路预测边集抽样时采用基于最小生成树的方式来简化抽样过程。
	\item 结合提出的增量图表征算法和增量节点属性表征算法应用于属性网络,在真实网络数据集中进行节点分类与链路预测与业界已有算法进行对比,在保证准确率的同时,能显著降低算法运行的时间成本。

\end{itemize}
\section{展望}
本文提出增量场景下的图表征算法,在真实数据集上表现出良好的准确率和时间效率。然而本文依然存在以下一些不足:
\begin{itemize}
	\item 本文的增量图表征算法iHLE基于拉普拉斯特征映射,拉普拉斯映射对于非全连通图无法进行有效表征。实际应用场景下的非连通网络结构是非常普遍的,因此需要优化并提出应用对非全连通网络上有效的增量表征方式。
	\item 本文基于属性网络的图表征算法SR1复杂度过高,除了可以通过利用结构信息进行优化以外,也可采用分布式的方法进行运行效率优化,分布式下可以在保留足够信息的同时优化运行速度。在速率得到充分优化的前提下,可以尝试采用高阶接近度对SR算法进行进一步优化。 
\end{itemize}
关于图表征学习的研究仍然是具有丰富应用场景的话题,本文的下一步工作可以从以下几个方面进行展开:
\begin{itemize}
	\item 本文研究对象为同构网络,也即网络中全部节点代表着相同种类的实体。现实应用中存在非常多的异构网络,比如网页链接,可能包含文本图像等各种不同的节点,对于异构网络中增量图表征学习的研究具有不错的应用需求。
	\item 本文研究内容为节点表征,在现实场景中,网络中边的表征甚至元路径(meta-path)的表征也具有非常广泛的应用场景,比如金融风控中对连边进行表征,从而实现对连边关系的风险识别等。因此可以将图表征对节点向量学习扩展到对连边等更高维网络成分的增量表征学习。
	\item 现实网络应用中存在节点数非常巨大的网络,直接进行全图学习是比较困难的,提出有效的分布式增量学习算法是非常具有应用前景的。
	\item 图表征算法是应用于机器学习任务的前置流程,增量图表征算法对应的需要有增量的机器学习框架,提出一个流程化的增量学习框架在现实场景中是非常必要的。
\end{itemize}