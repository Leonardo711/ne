\chapter{总结与展望}

\section{工作总结}
在移动互联网地快速发展下,移动应用市场上出现了海量移动应用。对于移动应用开发者而言,要想在如此众多的移动应用中脱颖而出充满了挑战。本文对最大的Android移动应用市场Google Play上的移动应用进行分析,发现评分较高、质量较好的移动应用倾向于使用第三方库。由于第三方库具有高封装性、可重用性和低维护成本等特点,可以协助开发者快速、高质地开发移动应用。为此,本文就这一问题展开研究,提出了基于文本和视图的TV-LibRec推荐算法,为开发者推荐优质、合适的第三方库。

本文从移动应用文本和视图两个层面,分析并提取了能够表征移动应用功能信息的特征。对于文本特征,本文重点分析移动应用元数据中的描述文字,对其进行数据清洗,并利用doc2vec模型将处理后的文本文字转化成有限维的特征向量。在文本特征向量基础上,本文提出了文本相似性计算方法,分别通过计算两个向量间的皮尔森相关系数和余弦相似度,从而构建移动应用在文本表示上的相似性。本文又从视图角度对移动应用进行功能分析,利用移动应用中表征视图信息的resources和assets资源数据,从程序包的源代码中搜寻相应的函数调用,从而构建移动应用在视图上的特征表示。此外,本文同样将移动应用的permissions权限申请作为特征,加入至视图特征向量中。通过计算两个视图向量间的Jaccard系数,本文构建了移动应用在视图表示上的相似性。

在移动应用相似性基础上,本文利用了聚类算法的优势,将相似的移动应用聚集到一起,从而提高第三方库搜寻的效率和精度。本文应用著名的聚类算法对移动应用进行聚类,并讨论分析了其在移动应用聚类中的局限性。为提高聚类算法的可靠性,本文提出了适用于移动应用聚类的k-min聚类算法,该算法能够充分利用移动应用两两间的相似性关系,从而提高聚类的准确度和稳定性。本文在聚类所得的移动应用簇基础上,为开发者搜寻最符合其功能需求的相近移动应用。在对相似移动应用筛选过程中,本文充分利用了移动应用间的相似性以及移动应用的用户评分,从而同时保证了移动应用的功能和质量,进而保证了筛选出的第三方库具有相同的性质。本文根据第三方库的推荐分数,对其进行排序,然后筛选一定量的第三方库,推荐给开发者使用。

本文从Google Play上爬取了23,398个移动应用的元数据和程序包,并从元数据中抽取本文所需的描述文本和用户评分,且对程序包进行逆向工程,转化成可读性较强的代码形式。在收集的数据集上,本文开展了多组验证实验,分别从本文算法的特征、聚类等部分进行研究分析,最后将提出的TV-LibRec推荐算法与其他著名的推荐算法进行比较,实验结果证明了本文提出的算法在第三方库推荐问题中的有效性。



\section{展望}
本文提出的TV-LibRec推荐算法是基于移动应用间的相似性,因此在相似性计算过程中仍有待提高。虽然本文建立的移动应用数据库,提前将市场上已有的移动应用进行反编译和分析,但在此过程中仍然需要大量的时间和计算资源。本文计划在后续工作中,利用少量移动应用间的相似性,来估算其他移动应用间的相似性,以减少计算成本,从而提高移动应用数据库构建过程中的资源消耗。此外,本文在视图相似性计算过程中暂时只考虑了视图元素,计划在未来的工作中对视图元素间的位置关系加以利用,以提高计算准确度。

移动应用聚类在本文提出的推荐算法中具有重要的地位。本文提出的k-min聚类算法,虽然具有较好的聚类效果,但需要人工设定聚类簇的个数,具有人为因素的影响。在未来的工作中,本文希望对该参数进行自动化,减少推荐算法中的人为干扰。同时,本文将研究并提出其他无参数的聚类算法,以提高聚类的效果和效率。

本文使用了白名单的方式提取移动应用中的第三方库,虽然能过获得较好的实验效果,但仍然会忽视少量第三方库。因此,在未来的工作中,本文将对第三方库检测方法进行研究,通过分析不同第三方库的特征信息,从而能够从移动应用程序包中快速识别出第三方库。

此外,本文将在未来工作中,与移动应用开发者进行讨论沟通,从而了解其在开发过程中遇到的问题和挑战。本文希望通过与开发者的交流,更加深入地了解开发者的需求,进而为开发者提供更多的服务和支持,以提高其开发效率和项目质量,为开发者创造更多的机会。