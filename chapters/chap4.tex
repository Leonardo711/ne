%%%%%%%%%%%%%%%%%%%%%%%%%%%%%%%%%%%%%%%
%----------------------------------------     算法     ---------------------------------------%
%%%%%%%%%%%%%%%%%%%%%%%%%%%%%%%%%%%%%%%
\chapter{属性网络中的增量图表征算法}
\section{引言}
图表征算法今年来吸引了大量关注,在图表征算法中一种朴素的思想是保留网络中节点的接近度,在上一章中本文提出了一种保留高阶接近度的网络表征学习算法,并相应地提出了增量场景下的图表征算法。可是,上一章的算法忽略了真实网络的特性,即网络中节点本身具有丰富的节点属性。网络除了拓扑结构信息以外,节点本身的属性也会影响影响节点分类和链路预测等的结果,比如,社交网络两个没有连接的用户节点,如果拥有越多共同的兴趣爱好或话题,那么成为好友或互相关注的可能性就会越大,因为用户节点除了通过物理连接形成在线社区以外,还会分享观点并评论而形成一些话题圈;相反没有共同兴趣爱好或话题但存在连接的用户节点则会使关系强度越变越弱。从现实的应用中可以看出,在有节点属性信息的网络中(后文将统一称为属性网络),节点的属性信息扮演着非常重要的角色。

在属性网络中,直接将网络中节点的属性作为节点表征向量的附加特征,再将组合得到的表征向量作为机器学习任务的输入是可行的,但是存在两个问题:1.对于节点特征为稀疏高维特征时,直接附加特征进行组合的方法会使得学习任务复杂度变高;2.忽略了网络结构与节点属性之间的耦合信息,节点属性特征之间的相关性被忽略。因此,不同于普通的网络图表征算法,属性网络的图表征算法需要平衡好网络结构相似度和节点属性之间的关联\cite{huang2017label}。

本章将提出一种属性网络中的图表征算法框架,以节点属性特征为主体,提出基于相似度的属性表征方法,根据分析对比模型的变化,提出增量场景下的相似度降维方法。

\section{问题描述}
假设一个无向属性网络$G=\{\textbf{A}, \textbf{H}\}$,其中\textbf{A}为网络的邻接矩阵,\textbf{H}为网络中节点的属性矩阵,且属性矩阵 $\textbf{H}\in R^{|V|\times d}$,$d$为节点属性特征的维数。属性网络图表征算法可视为寻找一个映射函数来表征网络中的每个节点:
\begin{equation}
f_G: \textbf{A}, \textbf{H} \rightarrow \textbf{X} \in R^{|V| \times k} \qquad s.t.\quad k<<|V|
\end{equation}

区别于上一章对静态场景和增量场景下网络结构的表征学习,本章研究对象为节点属性特征,基于节点属性的相似度对属性网络进行表征学习。下面将分别对静态场景和增量场景下的的属性网络表征学习进行介绍和分析。

\section{静态场景下属性网络表征学习}
在属性网络中对节点属性的表征学习不同于对节点属性进行直接的数据降维,前者更加关注节点属性之间的相似度关系,同时网络结构本身的信息可以对节点属性表征过程进行修改和调整,在这个过程中一个重要过程是通过节点属性特征计算相似度,下面将介绍三种相似度计算方法:
\definition{余弦相似度}
\definition{皮尔逊相似度}
\definition{KL散度}

三种相似度各有不同的优势和劣势,通过相似度计算可以得到节点属性特征的相似度矩阵$\textbf{S}^H$,基于相似度矩阵提出表征算法。
不同于节点邻接矩阵或其他相似度矩阵,属性特征的相似度矩阵不是稀疏矩阵,在此提出一种基于图分解的方式进行属性表征学习
\subsection{基于图分解的相似度表征学习}
介绍图分解

介绍更改的相似度表征学习方法流程

算法分析

\section{增量场景下属性网络表征学习}
%%%%%%%%%%%%%%%%%%%%%%%%%%%%%%%%%%%%%%%
%----------------------------------------     本章小结     ---------------------------------------%
%%%%%%%%%%%%%%%%%%%%%%%%%%%%%%%%%%%%%%%
\section{本章小结}
