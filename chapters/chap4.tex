%%%%%%%%%%%%%%%%%%%%%%%%%%%%%%%%%%%%%%%
%----------------------------------------     算法     ---------------------------------------%
%%%%%%%%%%%%%%%%%%%%%%%%%%%%%%%%%%%%%%%
\chapter{属性网络中的增量图表征算法}
\section{引言}
图表征算法今年来吸引了大量关注,在图表征算法中一种朴素的思想是保留网络中节点的接近度,在上一章中本文提出了一种保留高阶接近度的网络表征学习算法,并相应地提出了增量场景下的图表征算法。可是,上一章的算法忽略了真实网络的特性,即网络中节点本身具有丰富的节点属性。网络除了拓扑结构信息以外,节点本身的属性也会影响影响节点分类和链路预测等的结果,比如,社交网络两个没有连接的用户节点,如果拥有越多共同的兴趣爱好或话题,那么成为好友或互相关注的可能性就会越大,因为用户节点除了通过物理连接形成在线社区以外,还会分享观点并评论而形成一些话题圈;相反没有共同兴趣爱好或话题但存在连接的用户节点则会使关系强度越变越弱。从现实的应用中可以看出,在有节点属性信息的网络中(后文将统一称为属性网络),节点的属性信息扮演着非常重要的角色。

在属性网络中,直接将网络中节点的属性作为节点表征向量的附加特征,再将组合得到的表征向量作为机器学习任务的输入是可行的,但是存在两个问题:1.对于节点特征为稀疏高维特征时,直接附加特征进行组合的方法会使得学习任务复杂度变高;2.忽略了网络结构与节点属性之间的耦合信息,节点属性特征之间的相关性被忽略。因此,不同于普通的网络图表征算法,属性网络的图表征算法需要平衡好网络结构相似度和节点属性之间的关联\cite{huang2017label}。

本章将提出一种属性网络中的图表征算法框架,以节点属性特征为主体,提出基于相似度的属性表征方法,根据分析对比模型的变化,提出增量场景下的相似度降维方法。

\section{问题描述}
假设一个无向属性网络$G=\{\textbf{A}, \textbf{H}\}$,其中\textbf{A}为网络的邻接矩阵,\textbf{H}为网络中节点的属性矩阵,且属性矩阵 $\textbf{H}\in R^{|V|\times d}$,$d$为节点属性特征的维数。属性网络图表征算法可视为寻找一个映射函数来表征网络中的每个节点:
\begin{equation}
f_G: \textbf{A}, \textbf{H} \rightarrow \textbf{X} \in R^{|V| \times k} \qquad s.t.\quad k<<|V|
\end{equation}

区别于上一章对静态场景和增量场景下网络结构的表征学习,本章研究对象为节点属性特征,基于节点属性的相似度对属性网络进行表征学习。下面将分别对静态场景和增量场景下的的属性网络表征学习进行介绍和分析。

\subsection{静态场景下属性网络表征学习}
在属性网络中对节点属性的表征学习不同于对节点属性进行直接的数据降维,前者更加关注节点属性之间的相似度关系,同时网络结构本身的信息可以对节点属性表征过程进行修改和调整,在这个过程中一个重要过程是通过节点属性特征计算相似度,下面将介绍三种相似度计算方法:
\definition{}

\subsection{增量场景下属性网络表征学习}
%%%%%%%%%%%%%%%%%%%%%%%%%%%%%%%%%%%%%%%
%----------------------------------------     本章小结     ---------------------------------------%
%%%%%%%%%%%%%%%%%%%%%%%%%%%%%%%%%%%%%%%
\section{本章小结}
本章对本文所研究的问题——为移动应用开发者推荐优质合适的第三方库——作了具体定义,明确了本文研究问题的目标。针对本文的研究问题,本章从移动应用开发流程的两个阶段,提出了切实有效的解决方案。本文所提出的TV-LibRec算法框架一共分为了三个部分:首先,从移动应用中提取了能表征其功能的特征,并以此来计算移动应用间的相似性;然后,利用移动应用间的相似性,对功能相近的移动应用进行聚类操作;最后,在聚类后的簇中搜寻最符合开发者需求的第三方库,进而推荐给开发者使用。

本章针对移动应用相似性计算问题,分别从文本和视图两个数据层面,提出了可行的计算方法。移动应用描述信息是开发者对所开发的移动应用的功能描述文字,其包含了移动应用最基本的功能信息。因此,本文利用描述文本中的相关信息,从中提取表征移动应用功能的特征向量。本章详细介绍了本文对描述文本数据的处理过程,并阐述了doc2vec模型从文本到特征向量的转化过程。在文本特征向量的基础上,本章给出了针对文本特征的两种相似性计算方法:皮尔森相关系数和余弦相似度。本文亦从移动应用视图角度对功能特征进行抽取。本章详细阐述了针对移动应用的逆向工程方法,从而获得移动应用程序包的程序代码数据。在此基础上,本章介绍了移动应用中两种与视图相关的资源文件:resources和assets。此两类资源结合代码层数据,充分表征了移动应用的视图特征。本章同时利用了Android系统中的权限管理架构,即permissions特征,来对移动应用在敏感资源使用上的特征进行描述。综合三类移动应用内部的资源特征,本章给出了基于Jaccard系数的相似性计算方法,用以表征移动应用在视图层面的相似性。

在相似移动应用基础上,本章详细阐述了对其的聚类算法。本章首先考察了常用聚类算法k-means和单连接的层次聚类算法在本文问题上的适用性,并对其目标函数作相应调整,以适用于移动应用的聚类。本章详细分析了k-means和单连接的层次聚类算法在本文研究中的问题,并对该问题提出了相应的解决方法。为此,本章提出了一种针对移动应用聚类问题的新聚类算法k-min算法,该算法充分利用了任意两个移动应用间的相似性,从而解决了k-means和单连接的层次聚类算法在移动应用聚类过程遇到的问题和挑战。

最后,本章就本文研究问题,给出了可行的第三方库推荐算法。在相似移动应用聚类的基础上,本章针对开发者所需的目标移动应用,定位其所在的移动应用簇。推荐算法充分利用了移动应用相似性所表征的功能特征,以及移动应用评分所表征的应用质量,筛选了一个候选移动应用列表。该列表中的移动应用都具备高相似性、高质量的特点,分析并抽取其所使用的第三方库,可以确定适合开发者需求的第三方库。因此,本章将候选列表中的移动应用所使用的第三方库一一分析,并结合其相似性,计算第三方库的推荐分数。分数较高的第三方库则推荐给开发者使用。

本章对本文研究的问题作了深入分析和探讨,提出了基于文本和视图的TV-LibRec第三方库推荐算法。TV-LibRec充分利用了移动应用在文本和视图两个层次的相似性,并优化了已有的聚类算法,从而为开发者推荐优质合适的第三方库。