移动互联网的发展引起了移动应用的大规模产生。海量的移动应用给开发者带来了前所未有的机遇和挑战。对于移动应用开发者而言,要想在如此众多的应用中脱颖而出更是难上加难。移动应用质量、性能的好坏很大程度上受其使用的第三方库影响。使用优质、高封装的第三方库,可以有效加快开发进程,提高应用质量,增强用户体验。通过对现有移动应用市场上的移动应用进行统计学分析,本文发现评分较高的移动应用更倾向于使用第三方库。移动应用的评分是开发者最为关注的方面之一,因为评分的高低影响着移动应用的受欢迎程度,进而影响开发者的经济收益。为了协助移动应用开发者充分利用市场上已有的第三方库,本文提出TV-LibRec第三方库推荐算法。本文分别从移动应用的文本和视图两个层面,给出了两种不同的相似性计算方法,用以鉴定两个不同移动应用间的相似程度。在文本层面,本文利用移动应用元数据中的描述,对其进行分析和处理。本文先对描述文字进行数据清洗,将标点符号、链接、停用词等数据去除,然后将文字信息转化成向量形式,并使用皮尔森相关系数和余弦相似度计算不同描述间的相似性。在视图层面,本文充分利用Android移动应用中的两种视图数据资源:resources和assets,并从相应的程序代码中抽取特征,结合移动应用的权限申请,构成视图特征向量。本文使用Jaccard系数计算移动应用间的视图相似性。根据计算所得的文本和视图相似性值,本文提出了适用于移动应用聚类的k-min聚类算法,此聚类算法能够充分利用移动应用间的两两关系,从而保证簇内的移动应用最为相似,有效地将相似移动应用进行聚类,优于现有的聚类算法。最后,基于聚类后结果,本文充分利用移动应用的功能特征和质量情况,同时考虑相似移动应用所使用的第三方库以及其在移动应用市场上的评分,为开发者推荐优质合适的第三方库。本文从最大的Android移动应用市场Google Play上爬取了23,398个真实的移动应用,并在此数据集上进行全面的实验验证。实验结果表明,本文所提出的方法优于已有的其他著名推荐方法,能够有效地为移动应用开发者推荐所需的第三方库。