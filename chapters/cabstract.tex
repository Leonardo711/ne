随着社交网络的发展,网络数据的规模越来越庞大。海量的网络型数据对服务提供者包含着前所未有的机遇和挑战。对于服务提供者而言,可以利用对海量网络型数据的挖掘指导业务发展,如社交网络中通过网络的挖掘研究,可以进行高效的好友推荐;金融服务可以通过有效的挖掘任务规避用户在包括信用卡,借贷等信用业务中逾期甚至欺诈的风险。在机遇的同时,海量的网络型数据更带来巨大的挑战:一方面,对于海量的网络型数据直接进行数据处理和分析存储和计算成本非常巨大;另一方面,真实的网络数据是处在一直演化变动的过程中的,每当新增节点或连边就重新进行网络挖掘任务成本更是非常巨大。因此在网络挖掘中如何进行有效的网络信息抽取是一个非常值得研究的课题。在海量数据处理研究方面,用来提取数据重要信息的手段被称之为数据降维。受数据降维的启发,在近几年研究领域中网络挖掘领域的图表征算法引起非常大的关注。

本文的研究对象是现实场景中最常见的属性网络,也即网络中的节点除了网络连接结构以外,也包含大量的节点特征,例如社交网络中用户节点包含丰富的属性信息:账号、年龄、兴趣爱好等,对于节点属性的利用将更有利于网络型数据的学习任务。同时,属性网络的另一个特点也是本文的研究内容,即网络的动态演化。本文就动态演化的属性网络进行挖掘任务,对于网络结构提出一种HLE算法进行网络表征,通过相似度计算保留网络结构上的高阶接近度,也即保留网络中大于一度邻居的局部信息;基于HLE算法提出相应的增量表征算法iHLE来处理动态场景下的网络表征;对于节点属性提出一种新的算法SR进行属性表征学习,节点属性不同于网络结构的稀疏性,提出适合梯度下降的方式进行数据降维;同时提出相应的增量降维学习算法iSR对动态网络进行表征学习。

实验结果表明,本文提出的网络表征方式和增量表征方式在后续的节点分类和链路预测任务中都表现出理想的准确率和很好的时间效率。